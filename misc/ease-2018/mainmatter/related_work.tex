\section{Related work}

Given the incline of deep-learning approaches within the computer vision community, the interest in using Convolutional Neural Networks (CNNs)~\citep{Lecun:1998hy}---as spearheaded in \citep{Krizhevsky:2012wl}---in preference to traditional classifiers has grown \citep{Girshick:2014jx, Girshick:2015vr, Li:2016uj, Ren:2017ug, He:2017ud}, with use in a multitude of applications such as object recognition, text detection and facial recognition. The increasing ubiquity of these Machine Learning (ML) approaches emphasise the ever-growing need for software practitioners  to produce more training datasets and frameworks to train models and make inferences for predictions, complimentary skillsets for software engineering and data science teams, and thus more transitional processes from migrating cutting-edge ML research into tangible products.

Our production face recognition system, Theia, uses the ResNet-34 CNN \citep{He:2016ib}. The underlying architecture, \textit{FaceNet} \citep{Schroff:2015ks}, builds upon a mix of two architectures: (1) the Inception `GoogLeNet' \citep{Szegedy:2014tb} and (2) the \citet{Zeiler:2013ws} architectures (with additional layer modifications inspired by \citep{Lin:2013wb}). A training dataset of 260 million images reported an accuracy of 86.2\%. This is distinguished from previous methods that use a bottleneck layer or post-processing of results into other models or SVM classifiers \citep{Szegedy:2014tb,Taigman:2014vs}.

For use in production, a Python-based implementation of FaceNet was developed for use in Tensorflow. It is based on the OpenFace library \citep{amos2016openface} and borrows additional concepts sourced from \citep{Parkhi:2015de,Wen:2016jx}, as well as facial landmark detectors methods from Multi-Task CNN \citep{Zhang:2016wb}.
 
\begin{itemize}
  \item \textbf{Case studies} of ML implementation from pure research.
  \item Look at Google, Facebook, MS Research.
  \item Look at the `trace' of our \textbf{implemented work} in Theia.
  \item Trace back paper of FaceNet and see chain of prior research (``builds upon the work of XYZ'') and see which initial implementations there were (Step 2) to see varying implementations.
  \item Could do the same for multi-task CNN.
\end{itemize}