\section{Introduction}

Deep learning has taken off in recent years leading to the creation of many new models created by the research community. The field of computer vision in particular has gained significantly from deep learning \citep{He:2016ib,Jia:2014cm} especially in the areas of face recognition \citep{Schroff:2015ks} and object detection \citep{Huang:2017wn}. While researchers continue to improve the models and provide new capabilities, industry is left to play catch-up struggling to find ways to include these new advancements in products and services. Additional work is often needed to transform academic research into tangible outcomes suitable for industry.

\paragraph{Objective} To present experience from industry in incorporating state of the art Deep Learning models in a new product.

\paragraph{Methodology} This paper presents a case study of the ResNet-34 Convolutional Neural Network  (CNN) from research paper to inclusion in Theia, a production face recognition system processing 1.2 million images. We accompany the case study with insights gained from our practical experience of dealing with a deep learning tookit \citep{King:2009wp} and the challenges with providing evidence that the black box algorithms work as stated in particular with analysing training datasets \citep{Parkhi:2015de} and \citep{Ng:2014wa}.
 
 \paragraph{Contribution} The two main contributions of this paper are (1) a set of challenges facing industry when incorporating state of the art technology in new products and (2) a proposal for a meta-model to facilitate the transfer of research to industry based on Empirical Software Engineering principles.
 
\begin{itemize}
  \item A gap between the `theoretical' research and implementing this into `practical' product.
  \item The transition from research into an actual product.
  \item \textbf{Four-step process from pre-research in NN to the `accessible' layer}: pre-research; NN architecture paper + initial implementation; commercial implementation (industry/production grade); accessible layer (DSL that is user friendly).
  \item Various roles for each `step': Scientist/Researcher; Postdoc/PhD; Library Maintainer Engineer (Domain Specific); General Developer
  \item Thus, various skillsets needed. What are the gaps?
  \item In addition, need the model and data to make it work: need all three.
\end{itemize}